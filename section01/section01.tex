\documentclass[../main/main.tex]{subfiles}

\begin{document}

\section{Линейные радиотехнические цепи.}

\subsection{Условие квазистационарности.}
В общем случае электрические сигналы, проходя по цепям, изменяются во времени: 
$$i = i(t),~u=u(t),~\Phi = \Phi(t)~\text{и т.д.}$$

В окружающем пространстве имеется электромагнитное поле в виде электромагнитной волны, возникающей при переменных токах, зарядах и др. Волны несут информацию об "изменениях"\ в соседних точках цепи, что описывается функциями вида
$$f = f(t - \frac{x}{v}),$$
где $v$ --- скорость электромагнитной волны в данной среде, $x$ --- пространственная координата. Пусть $\tau_0$ --- характерное время изменения сигнала. Тогда, если $x~\ll~v\tau_0$ ($0\leqslant x \leqslant L$), во всех точках приближенно можно считать функцию $f(t, x)$ одинаковой, т.е. независимой в данный момент момент $t$ от координаты $x$: 
$$f(t, x) \bigg|_{x=0} \simeq f(t, x)\bigg|_{x=vt}\simeq f(t).$$

Другими словами, мы пренебрегаем эффектами запаздывания. Если $L$ --- характерный размер цепи, тогда потребуется:
$$x_{max} = L \ll v\tau_0 = \lambda~\text{(для гармонического сигнала: }~\tau_0=\frac{2\pi}{\omega})$$
и "условие квазистационарности": 
\begin{center}
$\boxed{L \ll \lambda}$ или $\frac{\tau}{\tau_0} \ll 1$,
\end{center}
где $\tau$ --- время передачи информации.

Цепи, удовлетворяющие этому условию, называются \underline{сосредоточенными цепями}. Пример: $\nu = 50$ Гц, $\lambda = c/\nu \simeq 6\cdot10^3$ км. Любая более короткая линия может считаться "сосредоточенной". Это, конечно, следствие низкой частоты.

\subsection{Линейные элементы цепей.}

Общим свойством простых сосредоточенных цепей является "линейность", т.е. подчинение принципу суперпозиции: реакция цепи на суммарный сигнал равна сумме реакций на каждый из сигналов в отдельности. Элементами таких цепей будут: сопротивление (резисторы), емкость (конденсаторы), индуктивность (катушки).

\begin{enumerate}
    \item Сопротивление $R$. % картиночка
    
    Также вводится понятие проводимости $G = \frac{1}{R}$. Размерности $R = [\text{Ом}],~G = [\text{Ом}^{-1}]$. Связь тока, напряжения и сопротивления (закон Ома): $i=\frac{U}{R}=GU$. $P=Ui$ --- мощность. $\Delta W_R = \int\limits_{0}^{t} Ui dt$ --- энергия, выделяемая на резисторе за время от $0$ до $t$.
    
    По отношению к реальным сопротивлениям --- это идеализация. Предполагается, что нет зависимостей $R(i)$ или $R(U)$ (в ином случае можно говорить о локальных $R$ и $G$ в окрестности точки $i = const$, их называют "дифференциальными"\ характеристиками: $R_\partial = \frac{dU}{di}\bigg|_{i=0},~G_\partial = R_\partial^{-1}$). На практике зависимость $R(i)$ может возникнуть, благодаря температурной вариации сопротивления $R(T)$: рост тока сопровождается нагревом и изменением $R$. Обычно на резисторе указывается предельно допустимая мощность, ниже которой "линейность"\ c заданной точностью гарантируется.
    
    \item Ёмкость $C$. 
    
    Заряд $q = CU_C$. "Ток смещения": $i = \frac{dq}{dt} = C \frac{dU_C}{dt}$. Энергия, выделяемая за время от $0$ до $t$: $W_C = \frac{1}{C} \int\limits_{0}^{t} i dt$. Вариация энергии: $\Delta W_C = W_C(t) - W(0) = \frac{C}{2} [U_C^2(t) - U_C^2(0)]$.
    
    Здесь также предполагается $C = const$, т.е. нет зависимости $C(U)$. Примеры, когда это не выполняется: конденсатор с сегнетоэлектриком, pn-переход и др. Если $C = C(U)$, то при $U = U_0 = const$ вводят $C_\partial = \frac{dq}{dU}$. 
    
    $\varepsilon = \frac{\varepsilon(0)}{1 + \bigg(\frac{\varepsilon(0)}{4\pi}\bigg)^3 BE^2}$ --- постоянная материала, где $B = const$, $E$ --- электрическое поле. Тогда, $C_U = \frac{C(0)}{1 + bU^2}$. В линейных системах эти эффекты опускаются. 
    
    \item Индуктивность $L$.
    
    Магнитный поток $\Phi = Li$. Напряжение $U = \frac{d\Phi}{dt} = L \frac{di}{dt}$. Выделяемая энергия на элементе $W_L = \frac{1}{2} Li^2$. Вариация энергии $\Delta W_L = \frac{L}{2} [i^2(t) - i^2(0)]$.
    
    Для этого элемента также возможно $L = L(i)$. Например, для катушки с сердечником $\mu = \mu(i) \Rightarrow L_\partial = \frac{d\Phi}{di}$.
\end{enumerate}

\subsection{Источники энергии.}

Это тоже элементы радиотехнических цепей: постоянные(батареи, аккумуляторы), переменные(генераторы). Эквивалентная схема источника должна содержать его внутреннее сопротивление. 

Здесь известны два предельных случая ("две абстракции"\ или "идеализации"):

\begin{enumerate}
    \item Генератор тока (идеальный источник тока).
        Внутреннее сопротивление велико по сравнению с сопротивлением внешней цепи (нагрузки): $R_i \gg R (G_i \rightarrow 0)$. Тогда, $i = \frac{U}{R_i + R} \simeq \frac{U}{R_i} = const$ --- т.е. ток не зависит от $R$! (источник снабжает нагрузку фиксированным током). 
    
    \item Генератор напряжения (идеальный источник напряжения).
     Внутреннее сопротивление мало по сравнению с сопротивлением внешней цепи: $R_i \ll R (R_i \rightarrow 0)$. Тогда, $i \simeq \frac{U}{R}$ или $U \simeq iR = U_0$ (напряжение, создаваемое во внешней цепи не зависит от нагрузки).  
\end{enumerate}

Реальные источники только приближенно могут быть отнесены к одному из этих генераторов.

\subsection{Уравнения простейших линейных цепей.}

\subsection{Метод комплексных амплитуд (МКА).}

\subsection{Расчет цепей методом комплексных амплитуд.}

\subsection{Метод преобразования Лапласа. Расчёт переходных режимов.}

\subsection{Последовательный колебательный контур.}

\subsection{Параллельный колебательный контур.}

\subsection{Осциллятор в радиофизике.}

\subsection{Метод ММА --- в гармоническом приближении.}

\subsection{Линейные четырёхполюсники.}

\subsection{Связанные колебательные контуры.}
\end{document}