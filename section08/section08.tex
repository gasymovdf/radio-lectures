\documentclass[../main/main.tex]{subfiles}

\begin{document}
\section{Выделение сигнала из шума.}

\subsection{Общность задач фильтрации и обработки данных измерений.}

На практике требуется проводить измерения наблюдательных величин (сигналов) в условиях, когда они маскируются флуктуациями -- "шумовым фоном". 

В программе данного курса рассматриваются основные приёмы ("алгоритмы фильтрации"), которые используются для улучшения отношения SNR -- "сигнал/шум".

\subsection{Повышение точности методом простого накопления.}

%Нужен рисунок!
Пусть "на выходе"{} имеется аддитивная смесь сигнала и шума: 

$|S|$ -- модуль сигнала, $\langle \xi^2 \rangle$ -- дисперсия шума.

В однократном измерении условием измерения будет: $$|S|_{min} \geqslant \sqrt{\langle \xi^2 \rangle}$$
\begin{equation}
    \max(SNR)_1 \geqslant 1
\end{equation}
Представим ситуацию, когда возможно повторение измерений на времени наблюдения $\tau_{\text{набл}}$. Тогда, сложение результатов $n$ измерений из этого $\tau_{\text{набл}}$ даёт: 

Шум $\rightarrow$ $\langle \xi^2 \rangle_\Sigma = n \langle \xi^2 \rangle$ (складываются дисперсии для независимых измерений);

Сигнал $\rightarrow$ $|S|_\Sigma = n |S|$ (складываются амплитуды).

Значит, соотношение <<сигнал/шум>> равно: 
\begin{equation}
    \left( \frac{S}{N} \right)_\Sigma = \left[ \frac{n |S|}{\sqrt{n \langle \xi^2 \rangle}} \right] = \sqrt{n} \left( \frac{S}{N}\right)_1 .
\end{equation}
Получается, $SNR \sim \sqrt{n}$! Данное соотношение выполнимо при выполнении условий: 
\begin{enumerate}
    \item Независимость измерений: $\tau_{\text{кор}} (\xi) < \tau_{\text{изм}}$;
    \item Большая длительность сигнала: $\tau_{\text{набл}} \gg \tau_{\text{изм}}$.
\end{enumerate}

Максимально возможное n связано со временем наблюдения: $n = \frac{\tau_{\text{набл}}}{\tau_{\text{изм}}}$, т.е. "чувствительность"{} растёт $\sim \sqrt{\frac{\tau_{\text{набл}}}{\tau_{\text{изм}}}}; \tau_{\text{набл}} > \tau_{\text{кор}}.$
Получается, что сделать большой выигрыш трудно: при $\tau_{\text{изм}} \sim 1$ с выигрыш в $10^1$ раз получится за 100 секунд, но выигрыш в $\sim 3 \cdot 10^2$ раз получится за $\sim 10^5$ секунд, т.е. примерно уже за сутки! 

\subsection{Принцип априорной информации в задачах обнаружения.}

"Чем больше сведений о сигнале и шуме, тем легче обнаружить (измерить) сигнал".

Пример: Обнаружение гармонического сигнала на фоне белого шума с $N_0 = \text{const}$:

\begin{enumerate}
    \item Частота сигнала неизвестна. Условие обнаружения: $$(\Delta U_C)_0 \geqslant \sqrt{4 \kappa T R \Delta f_0},$$ где $\Delta f_0 \sim 10^8$ Гц (широкая полоса приёма);
    
    \item Частота лежит в узкой полосе $f \in f_0 \pm \Delta f_1,~ \Delta f_1 \ll f_0$. Условием обнаружения будет: $$(\Delta U_C)_1 \geqslant \sqrt{4 \kappa T R \Delta f_1} = (\Delta U_C)_0 \cdot \sqrt{\frac{\Delta f_1}{\Delta f_0}} \ll 1!$$
    
    \item Известна также фаза сигнала ($\varphi \ll 2\pi$). Тогда: $$(\Delta U_C)_2 \gtrsim \sqrt{4 \kappa T R \Delta f_1 \frac{\varphi}{2\pi}} = (\Delta U_C)_1 \cdot \sqrt{\frac{\varphi}{2\pi}} \ll 1,$$
    т.е. чувствительность растёт за счёт сужения полосы приёма, как следствие наличия априорной информации. ("Загадка 2-х приятелей")
\end{enumerate}

\subsection{Критерии оптимальной фильтрации.}

Общая постановка задачи: на входе приёмного устройства есть смесь "сигнал+шум". Как следует обработать эту смесь, чтобы отношение сигнала к шуму было максимально возможным?

Рассмотрим два типичных случая: 

% \begin{comment}
\subsection{Обнаружение заданного сигнала в белом шуме. Корреляционный приемник.}

\label{white noise}
%картинка с фильтром
Задача ОФ: Дан сигнал $S(t)$ известной формы, $n(t)$ -- белый шум, т.е. $n(t) \equiv N_0$. Найти $K(\jo)$, чтобы $(S/N)$ было максимальным.

Решение: В момент $t = t_0$ на выходе блока фильтрации имеем:
\begin{equation}
    S_{\text{вых}} (t_0) = \frac{1}{2 \pi} \intinf S(\jo) K(\jo) e^{\jo t_0} \, d\omega;
\end{equation}

\begin{equation}
    \sigma_{n}^2 (t_0) = \frac{1}{2 \pi} \intinf N(\omega) |K(\jo)|^2 \,d\omega,
\end{equation}

выходные сигнал и шум, соответственно. Тогда, для $SNR$ имеем: 
\begin{equation}
\begin{split}
    \frac{S_{\text{вых}}(t_0)}{\sqrt{\sigma_{n}^2}} = \frac{\frac{1}{2 \pi} \intinf S(\jo) K(\jo) e^{\jo t_0}\, d\omega}{\left(\frac{N_0}{2\pi}\right)^{1/2} \left(\intinf |K(\jo)|^2 \,d\omega\right)^{1/2}} \lesssim \frac{\left[\frac{1}{2 \pi} \intinf |S(\jo)|^2\, d\omega \frac{1}{2 \pi} \intinf |K(\jo)|^2 \,d\omega \right]^{1/2}}{\left(\frac{N_0}{2\pi}\right)^{1/2} \left(\intinf |K(\jo)|^2 \, d\omega\right)^{1/2}} = \sqrt{\frac{2E_s}{N_0}}.
\end{split}
\end{equation}
Здесь введена замена $S(\jo) = S(\omega) e^{j \theta_s (\omega)}$; $S(\omega) \equiv |S(\jo)|$, а также было использовано неравенство Шварца: 

\begin{equation}
    \left|\int\limits_{a}^{b} F_1(x) F_2(x)\, dx \right|^2 \leq \int\limits_{a}^{b} |F_1(x)|^2 \, dx \cdot \int\limits_{a}^{b} |F_2(x)|^2 \, dx,
\end{equation}

причём максимум $SNR$ отвечает знаку равенства, который имеет место только при 
\begin{equation}\label{8.7}
    F_1(x) = A F_2^*(x),~A=\text{const}.
\end{equation}

Это условие определяет вид (структуру) оптимального фильтра $K(\jo)$, т.е. с учётом \ref{8.7}: $$K(\jo) e^{\jo t_0} = A S^*(\jo) = A S(\omega) e^{-j\theta_s(\omega)}.$$
И окончательно получаем: 
\begin{equation}
K(\jo) = |K(\jo)| e^{\varphi_K(\omega)} \equiv A S(\omega) e^{-j(\theta_s(\omega) + \omega t_0)}.
\end{equation}

Такая запись означает, что:
\begin{enumerate}
    \item $|K(\omega)| = S(\omega)$ -- модуль передаточной функции ОФ совпадает с амплитудным спектром сигнала. Совпадение модуля $|K(\omega)|$ со спектром сигнала означает отбрасывание компонент шума "вне сигнала".
    
    \item $\varphi_K(\omega) = -[\theta_s(\omega) + \omega t_0]$ -- фаза передаточной функции имеет противоположный знак $\Longleftrightarrow$ повёрнута на $\pi$! Это означает компенсацию фаз отдельных гармоник сигнала и они все складываются, образуя пик на выходе ОФ в момент $t_0$.
\end{enumerate}

Найдём переходную (импульсную) характеристику ОФ. 
\begin{equation}
\begin{split}
    h(t) = \frac{1}{2\pi} \intinf K(\jo) e^{\jo t} \, d\omega = \frac{A}{2\pi} \intinf S^*(\jo) e^{\jo (t-t_0)} \, d\omega = \\ = \frac{A}{2\pi} \intinf S(\jo) e^{\jo (t_0-t)} \, d\omega = A S(t_0-t).
\end{split}
\end{equation} 
По "принципу причинности"{}: $h(t) = 0~\text{при}~t<0$. Следовательно, $S(t_0 - t) = 0~\text{для}~t_0~<~t$. Сигнал в виде пика возникает в момент $t_0$, который соответствует окончанию сигнала на входе.

Переходная характеристика ("импульсная функция") оказывается зеркальным отражением сигнала по временной оси.
%Нарисовать картинку!

Используя импульсную характеристику $h(t)$ сигнал на выходе оптимального фильтра в момент $t_0$ можно записать в форме (сигнал на входе $x(t) = s(t) + \xi(t)$):
\begin{equation}
    u_{\text{вых}} (t_0) = A \int\limits_0^T x(t) S'(t_0 -t)\, dt.
\end{equation}

Блок-схема приёма: 
%нужен рисунок схемы

Т.о. для корреляционного приёма (согласованный фильтр) требуется генератор опорного сигнала (априорная информация). 
%нужны слайды -- примеры

\subsection{Фильтрация на фоне окрашенного шума.}

Усложним задачу --- откажемся от условия "белого шума"{}, т.е. сигнал вида $x(t) = s(t) + n(t)$, где шум $n(t)$ имеет некоторую спектральную интенсивность $N(\omega)$, т.е. шум имеет спектральную окраску. Найдём решение методом "сведения к предыдущей задаче".
%Нужен рисунок
На входе к оптимальному фильтру вставляем "единичный блок"{}$\{K_1(\jo)\}$. Запишем содержательное тождество: 
\begin{equation}
    K_1(\jo) \cdot \frac{1}{K_1(\jo)} \equiv 1.
\end{equation}

%история с математиком
Вид $K_1(\jo)$ выбираем таким:  
\begin{equation}
    K_1(\jo) = \sqrt{\frac{N_0}{N(\omega)}}.
\end{equation}

Тогда, после 1-го вспомогательного блока (в точке $B$) шум становится "белым"{}, действительно: для $N_1(\omega)$ имеем:   
\begin{equation}
    N_1(\omega) = N(\omega) \left| K_1(\jo) \right|^2 \equiv N_0 = Const.
\end{equation}

Т.е. прошла операция "отбеливания"{} шума. Тогда, в точке $B$ задача превращается в разобранный пункт \ref{white noise}, где была разобрана фильтрация на фоне белого шума. Следует учесть, что сигнал в точке $B$ тоже изменился и стал равным $S_1(\jo)$:
\begin{equation}
    S_1(\jo) = s(\jo) K_1(\jo) \neq S(\jo).
\end{equation}

Тогда, оставшуюся часть фильтра нужно формировать как:
\begin{equation}
   \begin{split}
   \frac{1}{K_1(\jo)} K(\jo) = A S_1^*(\jo) e^{-\jo t_0},~\text{или для $K(\jo)$ имеем:}\\
   K(\jo) = A S_1^*(\jo) e^{-\jo t_0} \cdot K_1(\jo) = A S_1^*(\jo) \left| K_1(\jo) \right|^2 e^{-\jo t_0} = A\, \frac{S^*(\jo)}{N(\omega)}\, e^{-\jo t_0}
   \end{split}
\end{equation}

Пройдя такой фильтр, сигнал со спектром $S(\jo)$ получает на выходе отношение $SNR \rightarrow \mu$, определённое формулой согласованной линейной фильтрации: 
\begin{equation}
\label{matched filter}
    \mu = \int\limits_0^\infty \frac{|S(\jo)|^2}{N(\omega)}\, d\omega.
\end{equation}

Физический смысл фильтра \ref{matched filter}: он "давит"{} спектральные компоненты, где шум большой, и "пропускает"{} все компоненты в спектре сигнала, т.е. здесь эффективно используется априорная информация о шуме. В частном случае белого шума, согласованная фильтрация переходит в "корреляционный приёмник".

\subsection{Обнаружение на смеси белого и окрашенного шумов. Фильтр Винера.}
Пусть имеется смесь 
\begin{equation} \label{mixture of signal and 2 noises}
    x(t) = s(t) + n(t) + \xi(t), 
\end{equation}

где $s(t)$ --- сигнал, $n(t)$ --- окрашенный гауссов шум и $\xi(t)$ --- белый шум, т.е. $\langle \xi(t) \xi(t+\tau) \rangle = N_0 \delta(t)$ % мог неправильно поставить знак равно, т.к. в лекции нет знака

Физика процесса: "белый шум"{} это флуктуации приёмных устройств $\xi(t)$, шум $n(t)$ идёт вместе с сигналом, может \underline{мультипликативно}: как шум параметров сигнала, сигнал связи с КА $\rightarrow~Y(t) = A\sin{\omega t + \varphi(t)}$, где $A,~\omega,~\varphi$ --- флуктуирующие параметры. Сам сигнал $s(t)$ должен рассматриваться как случайный процесс.  Тогда, для решения \ref{mixture of signal and 2 noises} используется "Винеровская фильтрация".

Задача фильтра Винера вводится как --- "обнаружение случайного $s(t)$ на фоне другого случайного $n(t)$". 
%нужен рисунок
Соответственно, требуется найти $K_B(\jo)$. Рассмотрим оценку сигнала: 
\begin{equation}
    \widehat{S}(\jo) = K_B(\jo) \left[ S(\jo) + n(\jo) \right]
\end{equation}

При прохождении ВФ есть динамическая ошибка $\Delta_d(\omega)$ и флуктуационная ошибка $\Delta_n(\omega)$. А именно: 
\begin{equation}
    \begin{cases}
        \Delta_d(\omega) = S(\jo) - K_B(\jo) S(\jo) = \left[ 1 - K_B(\jo) \right] S(\jo)\\
        \Delta_n(\omega) = K_B(\jo) n(\jo)
    \end{cases}
\end{equation}

Дисперсии вычисляются интегрированием: 
\begin{equation}
    \begin{cases}
        \sigma_d(\omega) = \frac{1}{\pi} \int\limits_0^\infty |1 - K_B(\jo)|^2 N_s(\jo)\,d\omega~\rightarrow~ N_s(\jo) = |S(\jo)|^2,~\text{динамическая дисперсия};\\
        \sigma_n(\omega) = \frac{1}{\pi} \int\limits_0^\infty |K_B(\jo)|^2 N_n(\jo)\,d\omega, ~\text{флуктуационная дисперсия}.
    \end{cases}
\end{equation}

Здесь мы ввели $N_s(\jo),~N_n(\jo)$ --- спектральные интенсивности сигнала и шума. При отыскании \textit{transfer function} $K_B(\jo)$ винеровского фильтра (ВФ) представим:
\begin{equation}\label{8.one}
    K_B(\jo) = A(\omega) + j\,B(\omega)
\end{equation}
\begin{equation}
    \begin{cases}
        A(\omega) = Re\, K_B(\jo)\\
        B(\omega) = Im\, K_B(\jo)
    \end{cases}
\end{equation}

Суммарная дисперсия (ошибка) равна $\sigma^2 = \sigma_d^2 + \sigma_n^2$, т.е.: 
\begin{equation}\label{8.two}
    \sigma^2 = \frac{1}{\pi} \int\limits_0^\infty \left(|1 - K_B(\jo)|^2 N_s(\jo) + |K_B(\jo)|^2 N_n(\jo) \right)\,d\omega.
\end{equation}

Структура ВФ находится из \underline{условия минимума полной ошибки}, т.е. экстремума $\sigma$ по переменным $A(\omega)$, $B(\omega)$: 
\begin{equation}\label{8.three}
    \frac{\partial \sigma^2}{\partial A} = 0;~~\frac{\partial \sigma^2}{\partial B} = 0.
\end{equation}

Используя соотношения \ref{8.one}, \ref{8.two} и \ref{8.three} можно получить структуру ВФ:
\begin{equation}
\begin{cases}
    K_B(\jo) = A(\omega) = \frac{N_s(\omega)}{N_s(\omega) + N_n(\omega)}.\\
    B(\omega) = 0
\end{cases}
\end{equation}

Обрезается зона $N_n > N_s$! 
%Нужен рисунок графика

\subsection{Обобщённая согласованная фильтрация Винера-Хопфа.}

Вернёмся к начальной задаче \ref{mixture of signal and 2 noises} фильтрации: 
$$ x(t) = s(t) + n(t) + \xi(t).$$

Следуя правилу обработки заданного сигнала на гауссовом шумовом фоне имеем \textit{filtering transfer function}:
\begin{equation} \label{first kopt}
    K_{opt} = Const\cdot \frac{S^*(\jo) e^{-\jo t_0} }{N_n(\jo) + N_0}.
\end{equation}
\begin{equation} \label{second kopt}
    \text{Или}~K_{opt} = Const \left[1 - \frac{N_n}{N_n + N_0} \right] \frac{S^*(\jo) e^{-\jo t_0} }{N_0}.
\end{equation}

%нужен рисунок блок-схемы фильтрации
Блок-схема фильтрации: 

Видно, что: в зоне $N_n(\jo) \gg N_0$ имеем \textit{matched filter} для \ref{first kopt}, и в зоне $N_0 \gg N_n(\jo)$ имеем \textit{correlator} для \ref{second kopt}.

Вводится терминология: винеровская оптимальная фильтрация. Во временном представлении обобщённый согласованный фильтр при 
\begin{equation}
    \begin{cases}
        x(t) -~\text{вход},~ x(t_0) = V \\
        y(t) -~\text{выход},~ \kappa(\tau) -~\text{корреляционная функция шума}.
    \end{cases}
\end{equation}

Покажем алгоритм Винера-Хопфа:
\begin{equation}
    y(t_0) = \int\limits_0^T x(t)\cdot h(t_0 - t)\, dt,
\end{equation}

где h(t) --- решение интегрального уравнения. Тогда: 
\begin{equation}
    \int\limits_0^T \kappa(t-\tau) h(\tau)\, d\tau = S(t).
\end{equation}

Опорная функция $h(t)$ определяется как формой сигнала, так и корреляционной функцией шума.

\subsection{Статистическая стратегия обнаружения.}

Теория оптимальной фильтрации содержит большое количество статистических задач среди них задачи обнаружения, оценки параметров реконструкции формы сигналов на различном шумовом фоне и др.). Мы рассматриваи только задачу обнаружения: сигнал известен, --- надо только его зарегистрировать (сказать, есть он или нет).

\textbf{Стратегия "да"{}~---~"нет".} %(аналог с марафонским гонцом).

При этом, все фильтры, рассмотренные в лекции, приводят к искажению сигнала. Они трансформируют сигнал ток, чтобы с максимальной вероятностью правильно ответить на вопрос (yes/no, $\lambda = \{0, 1\}$)!

На выходе такого фильтра в некоторый момент $t_0$ появляется максимальный выброс --- $V$--"оптимальная наблюдаемая"{}, или "достаточная статистика" по которой выносится решение.

\begin{enumerate}
    \item Напомним сведения из мат. статистики для случайной величины $V$ с интегральным распределением $P$ и порогом $C$:
    \begin{enumerate}
        \item $\alpha$ --- ошибка 1-го рода, или "вероятность ложной тревоги"{}, "вероятность случая":
        \begin{equation}
            \alpha = P\{\left.V>c \right|_{S = 0} \}, \text{принимается "да"{},}~\lambda=1,~\text{хотя сигнал отсутствует};
        \end{equation}
        
        \item $\beta$ --- ошибка 2-го рода, или "вероятность пропуска":
        \begin{equation}
            \beta = P\{\left.V<c \right|_{S \neq 0} \}, \text{принимается "нет"{},}~\lambda=0,~\text{хотя сигнал есть}.
        \end{equation}
    \end{enumerate}
    
    \begin{itemize}
        \item Статистика (интегральное распределение) наблюдаемой V предполагается известной;
        \item Порог $c$ выбирается по априорной информации. Если её нет, то по стратегии Неймана-Пирсона.
    \end{itemize}
    
    \item Критерий Неймана-Пирсона.
    
    Схема приёма (обработки) слабых сигналов имеет вид: 
    \begin{equation}
        x(t) = \lambda s(t) + n(t)
    \end{equation}
    %нужна картинка блоков
    
    Схема работает на интервале наблюдения $t \in [0, T]$.
    \begin{description}
        \item[1-й шаг.] Выбирается $\alpha$ --- в соответствие с мерой "ответственности наблюдателя".
        \item[2-й шаг.] По $\alpha$ и известной статистике $P$ рассчитывается порог $c$!
        \item[3-й шаг.] По наблюдениям за превышением порога выносится решение по $\lambda$ и рассчитывается величина $\beta$.
    \end{description}
    
    В этой схеме $\alpha,~\beta,~c$ могут двигаться (выбираться). Часто критерием Н-П называют выбор, максимизирующий "вероятность правильного обнаружения":
    \begin{equation}
        D = 1 - \beta,~\text{при}~ \alpha=Const.
    \end{equation}
    
    Эта величина в математике названа "мощностью критерия". Как увеличить $D$?
    
    Ответ: надо правильно (оптимально) выбрать блок обработки $K(\jo)$!
    
    \textbf{Корректный результат любого физического эксперимента в лаборатории и космосе должен представляться в рамках данной логики.}
\end{enumerate}
% \end{comment}
\end{document}
