\documentclass[12pt,a4paper]{article}
\usepackage{../style/summaryStyle}

% unique setting for each conspectus
\hypersetup{
    pdftitle={Конспект лекций по радиофизике},
    pdfauthor={проф. Руденко~В.Н., снс Гусев~А.В.},
    pdfsubject={Астрометрия},
    pdfcreator={Гасымов~Д.},
    %pdfproducer={???},
    pdfkeywords={Радиофизика} {Конспект},
    colorlinks=false
}

\begin{document}
\begin{summarytitlepage}
{проф. Руденко~В.Н., снс Гусев~А.В.}
{Радиофизика}
{---}
% можно добавить картинку по радиофизике
\end{summarytitlepage}

%\thispagestyle{plain}
\subsection*{Дисклеймер}
\noindent{\color{red!75!black} \textbf{Черновая версия: возможно большое количество различного рода ошибок!}}

\medskip

\tableofcontents\newpage
\subfile{../section00/section00.tex}\newpage
\subfile{../section01/section01.tex}\newpage
\subfile{../section02/section02.tex}\newpage
\subfile{../section03/section03.tex}\newpage
\subfile{../section04/section04.tex}\newpage
\subfile{../section05/section05.tex}\newpage
\subfile{../section06/section06.tex}\newpage
\subfile{../section07/section07.tex}\newpage
\subfile{../section08/section08.tex}\newpage
\subfile{../section09/section09.tex}\newpage
\subfile{../section10/section10.tex}\newpage
\subfile{../sectionA0/sectionA0.tex}\newpage
\end{document}

